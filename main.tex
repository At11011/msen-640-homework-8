\input{./src/main.sty}

\begin{document}

\input{./src/titlepage.tex}

\pagebreak

\begin{enumerate}
  \item Consider the blank phase diagram below
    \includegraphics[width=0.6\textwidth]{./assets/fig_1.png}
    \begin{enumerate}[(a)]
      \item Given that the phases present are $\alpha, \beta,
        \gamma,$ and $L$, label
        all of the single phase and two phase
        regions. Note that $\alpha$ is the crystal structure of pure
        component 1 and $\beta$ is the crystal structure of
        pure component 2.
      \item Sketch a plausible phase diagram for the same system with
        $T$ vs. $a_2$ (activity of component 2)
        axes.
    \end{enumerate}

    \pagebreak

  \item Consider a 2D strip of silicon (conductivity $\sigma =
      \SI{4.35e-4}{\per(\ohm\meter)}$, Seebeck coefficient
    $S = \SI{440}{\micro\volt\per\kelvin})$.
    Using the expression for electrical current density
    $\bm{J_q} = -\sigma\Delta\phi - \sigma S\Delta T$:

    \begin{enumerate}
      \item Determine the temperature gradient needed to maintain a
        potential difference of \SI{8}{\volt} across the
        material, assuming that no net current is transferred.
      \item If the material is a $\SI{1}{cm} \times \SI{1}{\centi\meter}$
        panel (with the bottom-left corner at the origin) and \SI{8}{\volt}
        is maintained across the $x$-direction with potential function
        $\phi(x) = (\SI{8}{\volt\per\centi\meter})x$ and a temperature
        difference of \SI{10}{\kelvin} is maintained along the y-direction
        with temperature function $T(y) = \SI{300}{\kelvin} +
        (\SI{10}{\kelvin/\centi\meter})y$, determine the magnitude of the
        current density.
    \end{enumerate}

    \pagebreak

  \item Consider a very simple model of a pickering emulsion. A
    spherical drop of oil with radius $r_{\text{oil}} = \SI{50}{\micro\meter}$
    suspended in water; the interfacial free energy density of the
    oil-water interface is
    $\gamma_\text{oil-water} = \SI{0.05}{\joule\per\meter\squared}$.
    Now, microspherical particles of radius $r_\mu = \SI{1}{\micro\meter}$
    that are functionalized to have an
    interfacial energy density
    $\gamma_\text{$\mu$-oil} = \SI{0.01}{\joule\per\meter\squared}$
    for regions exposed to oil and
    $\gamma_{\text{$\mu$-water}} = \SI{0.02}{\joule\per\meter\squared}$ for
    regions exposed to water.

    \begin{enumerate}[(a)]
      \item If a microsphere is brought from the pure water region into the
        pure oil region (so that it is completely immersed in water and oil),
        compute the change in enthalpy of the system, assuming that
        the enthalpy is dominated by the interfacial cost (H ≈ γA).
      \item If a microsphere is brought from the pure water region to the
        oil-water interface, calculate the change in entropy,
        assuming that half of the particle’s surface is in the oil and
        the other half is in the water.
      \item What is the change in the enthalpy of the oil-in-water droplet if
        50\% of the droplet’s surface is covered with microspheres?
      \item Using the result of (c) determine the effective interfacial free
        energy density $\gamma_\text{eff} = H/A$ of the oil droplet at
        50\% coverage.
    \end{enumerate}

    \pagebreak

  \item Consider a pool of molten magnesium in an open furnace (so that there
    is a vapor phase above the liquid phase). The melt is held at a
    temperature of $T = \SI{800}{\celsius}$ and has the following properties:

    \begin{align*}
      &\text{molar volume (liquid):} V^L =
      \SI{15.283e-6}{\meter\cubed\per\mole} \\
      &\Delta H = \SI{127.4}{\kilo\joule\per\mole} \\
      T_{\ch{Mg}}^b= \SI{1087}{\celsius} \text{ at atmospheric pressure} \\
      \gamma_{\ch{Mg}}^L = \SI{721e-5}{\joule\per\centi\meter\squared}
    \end{align*}
    \begin{enumerate}[(a)]
      \item Calculate the vapor pressure over the pool of molten magnesium
        in the furnace. [Hint: consider
        that the liquid and vapor phases are in equilibrium]
      \item Calculate the vapor pressure of magnesium within a
        \SI{10}{\nano\meter}
        bubble that is produced by cavitation.
    \end{enumerate}
\end{enumerate}

\pagebreak
\section*{Supporting code:}
\inputminted{julia}{./calculations/src/calculations.jl}

\end{document}


\begin{document}

\input{./src/titlepage.tex}

\pagebreak

\begin{enumerate}
  \item Consider the blank phase diagram below
    \includegraphics[width=0.6\textwidth]{./assets/fig_1.png}
    \begin{enumerate}[(a)]
      \item Given that the phases present are $\alpha, \beta,
        \gamma,$ and $L$, label
        all of the single phase and two phase
        regions. Note that $\alpha$ is the crystal structure of pure
        component 1 and $\beta$ is the crystal structure of
        pure component 2.
      \item Sketch a plausible phase diagram for the same system with
        $T$ vs. $a_2$ (activity of component 2)
        axes.
    \end{enumerate}

    \pagebreak

  \item Consider a 2D strip of silicon (conductivity $\sigma =
      \SI{4.35e-4}{\per(\ohm\meter)}$, Seebeck coefficient
    $S = \SI{440}{\micro\volt\per\kelvin})$.
    Using the expression for electrical current density
    $\bm{J_q} = -\sigma\Delta\phi - \sigma S\Delta T$:

    \begin{enumerate}
      \item Determine the temperature gradient needed to maintain a
        potential difference of \SI{8}{\volt} across the
        material, assuming that no net current is transferred.
      \item If the material is a $\SI{1}{cm} \times \SI{1}{\centi\meter}$
        panel (with the bottom-left corner at the origin) and \SI{8}{\volt}
        is maintained across the $x$-direction with potential function
        $\phi(x) = (\SI{8}{\volt\per\centi\meter})x$ and a temperature
        difference of \SI{10}{\kelvin} is maintained along the y-direction
        with temperature function $T(y) = \SI{300}{\kelvin} +
        (\SI{10}{\kelvin/\centi\meter})y$, determine the magnitude of the
        current density.
    \end{enumerate}

    \pagebreak

  \item Consider a very simple model of a pickering emulsion. A
    spherical drop of oil with radius $r_{\text{oil}} = \SI{50}{\micro\meter}$
    suspended in water; the interfacial free energy density of the
    oil-water interface is
    $\gamma_\text{oil-water} = \SI{0.05}{\joule\per\meter\squared}$.
    Now, microspherical particles of radius $r_\mu = \SI{1}{\micro\meter}$
    that are functionalized to have an
    interfacial energy density
    $\gamma_\text{$\mu$-oil} = \SI{0.01}{\joule\per\meter\squared}$
    for regions exposed to oil and
    $\gamma_{\text{$\mu$-water}} = \SI{0.02}{\joule\per\meter\squared}$ for
    regions exposed to water.

    \begin{enumerate}[(a)]
      \item If a microsphere is brought from the pure water region into the
        pure oil region (so that it is completely immersed in water and oil),
        compute the change in enthalpy of the system, assuming that
        the enthalpy is dominated by the interfacial cost (H ≈ γA).
      \item If a microsphere is brought from the pure water region to the
        oil-water interface, calculate the change in entropy,
        assuming that half of the particle’s surface is in the oil and
        the other half is in the water.
      \item What is the change in the enthalpy of the oil-in-water droplet if
        50\% of the droplet’s surface is covered with microspheres?
      \item Using the result of (c) determine the effective interfacial free
        energy density $\gamma_\text{eff} = H/A$ of the oil droplet at
        50\% coverage.
    \end{enumerate}

    \pagebreak

  \item Consider a pool of molten magnesium in an open furnace (so that there
    is a vapor phase above the liquid phase). The melt is held at a
    temperature of $T = \SI{800}{\celsius}$ and has the following properties:

    \begin{align*}
      &\text{molar volume (liquid):} V^L =
      \SI{15.283e-6}{\meter\cubed\per\mole} \\
      &\Delta H = \SI{127.4}{\kilo\joule\per\mole} \\
      T_{\ch{Mg}}^b= \SI{1087}{\celsius} \text{ at atmospheric pressure} \\
      \gamma_{\ch{Mg}}^L = \SI{721e-5}{\joule\per\centi\meter\squared}
    \end{align*}
    \begin{enumerate}[(a)]
      \item Calculate the vapor pressure over the pool of molten magnesium
        in the furnace. [Hint: consider
        that the liquid and vapor phases are in equilibrium]
      \item Calculate the vapor pressure of magnesium within a
        \SI{10}{\nano\meter}
        bubble that is produced by cavitation.
    \end{enumerate}
\end{enumerate}

\pagebreak
\section*{Supporting code:}
\inputminted{julia}{./calculations/src/calculations.jl}

\end{document}


\begin{document}

\input{./src/titlepage.tex}

\pagebreak

\begin{enumerate}
  \item Consider the blank phase diagram below
    \includegraphics[width=0.6\textwidth]{./assets/fig_1.png}
    \begin{enumerate}[(a)]
      \item Given that the phases present are $\alpha, \beta,
        \gamma,$ and $L$, label
        all of the single phase and two phase
        regions. Note that $\alpha$ is the crystal structure of pure
        component 1 and $\beta$ is the crystal structure of
        pure component 2.
      \item Sketch a plausible phase diagram for the same system with
        $T$ vs. $a_2$ (activity of component 2)
        axes.
    \end{enumerate}

    \pagebreak

  \item Consider a 2D strip of silicon (conductivity $\sigma =
      \SI{4.35e-4}{\per(\ohm\meter)}$, Seebeck coefficient
    $S = \SI{440}{\micro\volt\per\kelvin})$.
    Using the expression for electrical current density
    $\bm{J_q} = -\sigma\Delta\phi - \sigma S\Delta T$:

    \begin{enumerate}
      \item Determine the temperature gradient needed to maintain a
        potential difference of \SI{8}{\volt} across the
        material, assuming that no net current is transferred.
      \item If the material is a $\SI{1}{cm} \times \SI{1}{\centi\meter}$
        panel (with the bottom-left corner at the origin) and \SI{8}{\volt}
        is maintained across the $x$-direction with potential function
        $\phi(x) = (\SI{8}{\volt\per\centi\meter})x$ and a temperature
        difference of \SI{10}{\kelvin} is maintained along the y-direction
        with temperature function $T(y) = \SI{300}{\kelvin} +
        (\SI{10}{\kelvin/\centi\meter})y$, determine the magnitude of the
        current density.
    \end{enumerate}

    \pagebreak

  \item Consider a very simple model of a pickering emulsion. A
    spherical drop of oil with radius $r_{\text{oil}} = \SI{50}{\micro\meter}$
    suspended in water; the interfacial free energy density of the
    oil-water interface is
    $\gamma_\text{oil-water} = \SI{0.05}{\joule\per\meter\squared}$.
    Now, microspherical particles of radius $r_\mu = \SI{1}{\micro\meter}$
    that are functionalized to have an
    interfacial energy density
    $\gamma_\text{$\mu$-oil} = \SI{0.01}{\joule\per\meter\squared}$
    for regions exposed to oil and
    $\gamma_{\text{$\mu$-water}} = \SI{0.02}{\joule\per\meter\squared}$ for
    regions exposed to water.

    \begin{enumerate}[(a)]
      \item If a microsphere is brought from the pure water region into the
        pure oil region (so that it is completely immersed in water and oil),
        compute the change in enthalpy of the system, assuming that
        the enthalpy is dominated by the interfacial cost (H ≈ γA).
      \item If a microsphere is brought from the pure water region to the
        oil-water interface, calculate the change in entropy,
        assuming that half of the particle’s surface is in the oil and
        the other half is in the water.
      \item What is the change in the enthalpy of the oil-in-water droplet if
        50\% of the droplet’s surface is covered with microspheres?
      \item Using the result of (c) determine the effective interfacial free
        energy density $\gamma_\text{eff} = H/A$ of the oil droplet at
        50\% coverage.
    \end{enumerate}

    \pagebreak

  \item Consider a pool of molten magnesium in an open furnace (so that there
    is a vapor phase above the liquid phase). The melt is held at a
    temperature of $T = \SI{800}{\celsius}$ and has the following properties:

    \begin{align*}
      &\text{molar volume (liquid):} V^L =
      \SI{15.283e-6}{\meter\cubed\per\mole} \\
      &\Delta H = \SI{127.4}{\kilo\joule\per\mole} \\
      T_{\ch{Mg}}^b= \SI{1087}{\celsius} \text{ at atmospheric pressure} \\
      \gamma_{\ch{Mg}}^L = \SI{721e-5}{\joule\per\centi\meter\squared}
    \end{align*}
    \begin{enumerate}[(a)]
      \item Calculate the vapor pressure over the pool of molten magnesium
        in the furnace. [Hint: consider
        that the liquid and vapor phases are in equilibrium]
      \item Calculate the vapor pressure of magnesium within a
        \SI{10}{\nano\meter}
        bubble that is produced by cavitation.
    \end{enumerate}
\end{enumerate}

\pagebreak
\section*{Supporting code:}
\inputminted{julia}{./calculations/src/calculations.jl}

\end{document}


\begin{document}

\input{./src/titlepage.tex}

\pagebreak

\begin{enumerate}
  \item Consider the blank phase diagram below
    
    \begin{figure}[h]
        \centering
      \includegraphics[width=0.6\textwidth]{./assets/fig_1.png}
  \end{figure}
    \begin{enumerate}[(a)]
      \item Given that the phases present are $\alpha, \beta,
        \gamma,$ and $L$, label
        all of the single phase and two phase
        regions. Note that $\alpha$ is the crystal structure of pure
        component 1 and $\beta$ is the crystal structure of
        pure component 2.

        \boxedanswer{
            \centering
            \includegraphics[width=0.5\textwidth]{./assets/fig_1_sol.png}
        }

      \item Sketch a plausible phase diagram for the same system with
        $T$ vs. $a_2$ (activity of component 2)
        axes.

    \boxedanswer{
    The two-phase regions collapse into a single line, resulting in only
    one-phase regions having any area.


       \centering
       \includegraphics[width=0.4\textwidth]{./assets/fig_2.png}
    }


    \end{enumerate}

    \pagebreak

  \item Consider a 2D strip of silicon (conductivity $\sigma =
      \SI{4.35e-4}{\per\ohm\per\meter}$, Seebeck coefficient
    $S = \SI{440}{\micro\volt\per\kelvin})$.
    Using the expression for electrical current density
    $\bm{J_q} = -\sigma\nabla\phi - \sigma S\nabla T$:

    \begin{enumerate}
      \item Determine the temperature difference needed to maintain a
        potential difference of \SI{8}{\micro\volt} across the
        material, assuming that no net current is transferred.

        \boxedanswer{
            \begin{align*}
                0 &= -\sigma\nabla\phi  - \sigma S\nabla T \\
                \sigma\nabla\phi &=  - \sigma S\nabla T \\
                \nabla T &= -\frac{\nabla \phi}{S} \\
                \nabla T &= -\frac{\SI{8e-6}{\volt}}{\SI{440e-6}{\volt\per\kelvin}} \\
                \Aboxed{\nabla T &= -\SI{0.018}{\kelvin}}
            \end{align*}

            The negative indicates that the voltage gradient is in the opposite
            direction as the temperature gradient.
        }

      \item If the material is a $\SI{1}{cm} \times \SI{1}{\centi\meter}$
        panel (with the bottom-left corner at the origin) and \SI{8}{\micro\volt}
        is maintained across the $x$-direction with potential function
        $\phi(x) = (\SI{8}{\volt\per\centi\meter})x$ and a temperature
        difference of \SI{10}{\kelvin} is maintained along the y-direction
        with temperature function $T(y) = \SI{300}{\kelvin} +
        (\SI{10}{\kelvin/\centi\meter})y$, determine the magnitude of the
        current density.

        \boxedanswer{
            \begin{align*}
                \phi(x) &= \SI{8}{\micro\volt\per\centi\meter}\cdot x \\
                \frac{d\phi}{dx} &= \frac{\SI{8e-6}{\volt}}{\SI{0.01}{\meter}} = 
                    \SI{8e-4}{\volt\per\meter} \\
                T(y) &= \SI{300}{\kelvin} + \SI{10}{\kelvin}\cdot y \\ 
                \frac{dT}{dy} &= \frac{\SI{10}{\kelvin}}{\SI{0.01}{\meter}} = \SI{1000}{\kelvin\per\meter} \\
                S &= \SI{400e-6}{\volt\per\kelvin} = \SI{4.40e-4}{\volt\per\kelvin} \\
                J &= -\sigma\nabla\phi - \sigma S \nabla T \\
                \nabla\phi &= \frac{d\phi}{dx}\hat i \\
                \nabla T &= \frac{dT}{dy}\hat j \\
                J_x &= -\sigma\frac{d\phi}{dx} = -\sigma(\SI{8e-4}{\ampere\per\meter\squared}) \\
                J_y &= -\sigma S\frac{dT}{dy} = -\sigma(\SI{0.44}{\ampere\per\meter\squared}) \\
                J &= -\sigma(\num{8e-4}\hat i + \num{0.44}\hat j)\si{\ampere\per\meter\squared} \\
                |J| &= \sigma\sqrt{(\num{8e-4})^2 + (\num{0.44})^2}\si{\ampere\per\meter\squared} \\
                |J| &= -\sigma(0.44)\si{\ampere\per\meter\squared} \\
                |J| &= -(\SI{4.35e-4}{\per\ohm\per\meter})(0.44) \\
                \Aboxed{|J| &= \SI{1.914e-4}{\ampere\per\meter\squared}}
            \end{align*}
        }

    \end{enumerate}

    \pagebreak

  \item Consider a very simple model of a pickering emulsion. A
    spherical drop of oil with radius $r_{\text{oil}} = \SI{50}{\micro\meter}$
    suspended in water; the interfacial free energy density of the
    oil-water interface is
    $\gamma_\text{oil-water} = \SI{0.05}{\joule\per\meter\squared}$.
    Now, microspherical particles of radius $r_\mu = \SI{1}{\micro\meter}$
    that are functionalized to have an
    interfacial energy density
    $\gamma_\text{$\mu$-oil} = \SI{0.01}{\joule\per\meter\squared}$
    for regions exposed to oil and
    $\gamma_{\text{$\mu$-water}} = \SI{0.02}{\joule\per\meter\squared}$ for
    regions exposed to water.

    \begin{enumerate}[(a)]
      \item If a microsphere is brought from the pure water region into the
        pure oil region (so that it is completely immersed in water and oil),
        compute the change in enthalpy of the system, assuming that
        the enthalpy is dominated by the interfacial cost ($H \approx \gamma A$).

        \boxedanswer{
            \begin{align*}
                r_{\text{oil}} &= \SI{50e-6}{\meter} \\
                r_{\mu} &= \SI{1e-6}{\meter} \\
                \gamma_{\text{oil--water}} &= \SI{0.05}{\joule\per\meter\squared} \\
                \gamma_{\mu\text{--oil}} &= \SI{0.01}{\joule\per\meter\squared} \\
                \gamma_{\mu\text{--water}} &= \SI{0.02}{\joule\per\meter\squared} \\
                A_\mu &= 4\pi r_\mu^2 = 4\pi(\SI{1e-6}{\meter})^2 = \SI{1.257e-11}{\meter\squared} \\
                A_\text{drop} &= 4\pi r_{\text{oil}}^2 = \SI{3.141e-8}{\meter\squared} \\
                \Delta H &= H_f - H_i = (\gamma_{\mu{\text{--oil}}} - \gamma_{\mu\text{--water}})A_\mu \\
                \Delta H &= (\SI{0.01}{\joule\per\meter\squared} - \SI{0.02}{\joule\per\meter\squared})(\SI{1.257e-11}{\meter\squared})  \\
                \Aboxed{\Delta H &= \SI{-1.257e-13}{\joule}}
            \end{align*}
        }

      \item If a microsphere is brought from the pure water region to the
        oil-water interface, calculate the change in enthalpy,
        assuming that half of the particle’s surface is in the oil and
        the other half is in the water.

        \boxedanswer{
            \begin{align*}
                H_\text{int} &= \frac{1}{2}\gamma_{\mu{\text{--oil}}}A_\mu + 
                    \frac{1}{2}\gamma_{\mu\text{--water}}A_\mu \\ 
                H_\text{int} &= \SI{1.885e-13}{\joule} \\ 
                H_i &= \gamma_{\mu\text{--water}}A_\mu \\
                H_i &= \SI{2.513e-13}{\joule} \\
                \Delta H &= H_\text{int} - H_i \\
                \Delta H &= \SI{1.885e-13}{\joule} - \SI{2.513e-13}{\joule} \\
                \Aboxed{\Delta H &= \SI{-6.28e-14}{\joule}}
            \end{align*}
        }

      \item What is the change in the enthalpy of the oil-in-water droplet if
        50\% of the droplet’s surface is covered with microspheres?

        \boxedanswer{
            \begin{align*}
                \Delta H &= \left[\overbrace{\frac{1}{2}\gamma_{\text{oil--water}} +
                    \frac{1}{2}\cdot\frac{1}{2}(\gamma_{\mu\text{--oil}}+
            \gamma_{\mu\text{--water}})}^{\gamma_\text{eff}} - \gamma_{\text{oil--water}}\right]A_\text{oil} \\
                \Aboxed{\Delta H &= \SI{-5.498e-10}{\joule}} \\
            \end{align*}
        }

    \pagebreak

      \item Using the result of (c) determine the effective interfacial free
        energy density $\gamma_\text{eff} = H/A$ of the oil droplet at
        50\% coverage.

        \boxedanswer{
            \begin{align*}
                \gamma_\text{eff} &= \frac{1}{2}\gamma_{\text{oil--water}} +
                    \frac{1}{2}\cdot\frac{1}{2}(\gamma_{\mu\text{--oil}}+
                    \gamma_{\mu\text{--water}}) \\
                \Aboxed{\gamma_\text{eff} &= \SI{0.0325}{\joule\per\meter\squared}}
            \end{align*}
        }

    \end{enumerate}

    \pagebreak

  \item Consider a pool of molten magnesium in a closed furnace (so that there
    is a vapor phase above the liquid phase). The melt is held at a
    temperature of $T = \SI{800}{\celsius}$ and has the following properties:

    \begin{align*}
      &\text{molar volume (liquid):} V^L =
      \SI{15.283e-6}{\meter\cubed\per\mole} \\
      &\Delta H = \SI{127.4}{\kilo\joule\per\mole} \\
      &T_{\ch{Mg}}^b= \SI{1087}{\celsius} \text{ at atmospheric pressure} \\
      &\gamma_{\ch{Mg}}^L = \SI{721e-5}{\joule\per\centi\meter\squared}
    \end{align*}

    \begin{enumerate}[(a)]
      \item Calculate the vapor pressure over the pool of molten magnesium
        in the furnace. [Hint: consider
        that the liquid and vapor phases are in equilibrium]

        \boxedanswer{
            \begin{align*}
                T &= \SI{1073.15}{\kelvin} \\
                T_b &= \SI{1360.15}{\kelvin} \\
                P_{\text{atm}} &= \SI{101325}{\pascal} \\
                \Delta H &= \SI{127400}{\joule\per\mole} \\
                d &= \SI{10e-9}{\meter} \\
                r &= \SI{5e-9}{\meter} \\
                \gamma &= \SI{0.721}{\joule\per\centi\meter\squared} \\
                \ln\left(\frac{P_{\text{flat}}}{P_{\text{atm}}}\right) &=
                    -\frac{\Delta H}{R}\left(\frac{1}{T}-\frac{1}{T_b}\right) \\
                P_{\text{flat}} &= P_{\text{atm}}\exp\left[-\frac{\Delta H}{R}
                    \left(\frac{1}{T}-\frac{1}{T_b}\right)\right] \\
                    P_{\text{flat}} &= (\SI{101325}{atm})\exp\left[
                        -\frac{(\SI{127400}{\joule\per\mole})}
                        {(\SI{8.314}{\joule\per\mole\per\kelvin})}
                    \left(\frac{1}{(\SI{1073.15}{\kelvin})}-
                    \frac{1}{\SI{1360.15}{\kelvin}}\right)\right] \\
                        \Aboxed{P_{\text{flat}} &= \SI{4980.53}{\pascal}}
            \end{align*}
        }

      \item Calculate the vapor pressure of magnesium within a
        \SI{10}{\nano\meter} diameter
        bubble that is produced by cavitation.

        \boxedanswer{
            \begin{align*}
                \ln\frac{p_r}{p} &= \frac{2\gamma V^L}{RTr} \\
                p_r &= p\exp\left[\frac{2\gamma V^L}{RTr}\right] \\
                p_r &= p\exp\left[\frac{2(\SI{0.721}{\joule\per\centi\meter\squared})
                        (\SI{15.283e-6}{\meter\cubed\per\mole})
                    \left(\frac{4}{3}\pi(\SI{5e-9}{\meter})^3\right)}
                {(\SI{8.314}{\joule\per\mole\per\kelvin})(\SI{1360.15}{\kelvin})(\SI{5e-9}{\meter})}\right] \\
                        \Aboxed{p_r &= \SI{5232.734}{\pascal}}
            \end{align*}
        }

    \end{enumerate}
\end{enumerate}

% \pagebreak
% \section*{Supporting code:}
% \inputminted{julia}{./calculations/src/calculations.jl}

\end{document}
